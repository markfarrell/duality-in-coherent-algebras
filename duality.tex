\documentclass[notheorems]{beamer}

\usetheme{Warsaw}
\useoutertheme{miniframes}
\useinnertheme{rectangles}
\usecolortheme{albatross}

\usepackage[utf8]{inputenc}

\usepackage{amsmath}
\usepackage{amsthm}

\theoremstyle{definition} % insert bellow all blocks you want in normal text
\newtheorem{definition}{Definition}[section] % to number according to section
\newtheorem{example}{Example}[section]
\newtheorem{proposition}{Proposition}[section]
\newtheorem{theorem}{Theorem}[section]

%Information to be included in the title page:
\title{Duality in Coherent Algebras}
\author{Mark Farrell}
\date{June 2018}

\begin{document}

\begin{frame}
\titlepage
\end{frame}

\begin{frame}

\frametitle{Coherent Algebras}

	\begin{definition}

		  A coherent algebra $\mathcal{W}$ of order $n$ is a subspace of $\operatorname{Mat}_{n \times n}(\mathbb{C})$ such that:

		  \begin{itemize}
		  	\item $M N$ and $M \circ N \in \mathcal{W}$ for all $M, N \in \mathcal{W}$.
		  	\item $M^{T} \in \mathcal{W}$ for all $M \in \mathcal{W}$.
		  	\item $I, J \in \mathcal{W}$.
		  \end{itemize}

  \end{definition}

\end{frame}

\begin{frame}

\frametitle{Primitive Matrices}

	\begin{definition}
		An element $M \in \mathcal{W}$ of a coherent algebra $\mathcal{W}$ is said to be primitive if $M^{2} = M$ and $M N \in \operatorname{span} \{ M \}$ for all $N \in \mathcal{W}$.
	\end{definition}

\end{frame}

\begin{frame}

\frametitle{Schur-Primitive Matrices}

	\begin{definition}
		Dually, an element $M \in \mathcal{W}$ of a coherent algebra $\mathcal{W}$ is said to be Schur-primitive if $M^{\circ 2} = M$ and $M \circ N \in \operatorname{span} \{ M \}$ for all $N \in \mathcal{W}$.
	\end{definition}

\end{frame}

\begin{frame}

\frametitle{Dual Bases}


	\begin{theorem}
	Every commutative coherent algebra is spanned by its set of primitive matrices relative to both Schur-product and ordinary matrix multiplication.
	\end{theorem}

\end{frame}

\begin{frame}



	Suppose that $\mathcal{W}$ is a commutative coherent algebra of order $n$.

	\vspace{1em}

	Let $\Gamma(\mathcal{W})$ denote the set of Schur-primitive matrices in $\mathcal{W}$.

	\vspace{1em}

	Dually, let $\Lambda(\mathcal{W})$ denote the set of primitive matrices in $\mathcal{W}$.

	\vspace{1em}

	Suppose that $\{ A_{i} : 1 \leq i \leq d \}$ and $\{ E_{i} : 1 \leq i \leq d \}$ are orderings 
	on $\Gamma(\mathcal{W})$ and $\Lambda(\mathcal{W})$, letting $d := \operatorname{dim}(\mathcal{W})$.

\end{frame}

\begin{frame}

\frametitle{Character Tables and Dual Character Tables}

	\only<1->{The matrix $P \in \operatorname{Mat}_{d \times d}(\mathbb{C})$ such that $A_{i} = \sum\limits_{j = 1}^{d} P_{i, j} E_{j}$ for all $1 \leq i \leq d$ 
	is said to be the character table of $\mathcal{W}$ (under these orderings).}

	\only<2->{
		\vspace{1em}

		Dually, the matrix $Q \in \operatorname{Mat}_{d \times d}(\mathbb{C})$ such that $E_{i} = \sum\limits_{j = 1}^{d} \left ( \frac{1}{n} Q_{i, j} \right ) A_{j} $  for all $1 \leq i \leq d$
		is said to be the dual character table of $\mathcal{W}$ (under these orderings).

	}

	\only<3->{
		\vspace{1em}

		We have that $P^{-1} = \frac{1}{n} Q$ and hence $P Q = n I$.
	}

\end{frame}

\begin{frame}

\frametitle{Dual Character Tables in Terms of Character Tables}

	\only<1->{
		Assume that $\mathcal{W}$ is symmetric.
	}

	\only<2->{
		\vspace{1em}

		Define the trace inner product $\langle \cdot, \cdot \rangle$ on $\operatorname{Mat}_{d \times d}(\mathbb{C})$:

		\vspace{1em}

		$\langle \cdot , \cdot \rangle : \operatorname{Mat}_{d \times d}(\mathbb{C}) \times \operatorname{Mat}_{d \times d}(\mathbb{C}) \to \mathbb{C}$

		\vspace{0.5em}

		$\langle A, B \rangle := \operatorname{tr} \left ( B^{*} A \right ) = \operatorname{sum} \left ( \overline{B} \circ A \right )$

	}

	\only<3->{

		\vspace{1.5em}

		Let $v_i$ denote the valency of $A_{i}$ and $m_{i}$ denote the rank of $E_{i}$

		\vspace{0.5em}

		for all $1 \leq i \leq d$.

	}

	\only<4->{

		\vspace{1em}

		$ \implies v_{j} Q_{i,j} = \langle E_{i}, A_{j} \rangle = m_{i} P_{j, i}$ for all $1 \leq i, j \leq d $.

	}

	\only<5->{

		\vspace{1em}

		Hence $Q = R \circ P^{T}$ where $R \in \operatorname{Mat}_{d \times d}(\mathbb{C})$ and $R_{i, j} := \frac{m_{i}}{v_{j}}$  for all $1 \leq i, j \leq d $.

	}

\end{frame}

\begin{frame}

\frametitle{Intersection Numbers and Dual Intersection Numbers}

	\only<1>{
		There is a set of numbers $\{ p_{i,j}(k) : 1 \leq i,j,k \leq d \}$ such that $A_{i} A_{j} = \sum\limits_{k = 1}^{d} p_{i, j}(k) A_{k}$ for all $ 1 \leq i, j \leq d$.

		\vspace{1em}

		These are said to be the intersection numbers of $\mathcal{W}$.
	}

	\only<2>{
		Dually, there is a set of numbers $\{ q_{i,j}(k) : 1 \leq i,j,k \leq d \}$  such that $E_{i} \circ E_{j} = \sum\limits_{k = 1}^{d} \left ( \frac{1}{n} q_{i, j}(k) \right ) E_{k}$ for all $ 1 \leq i, j \leq d$.

		\vspace{1em}

		These are said to be the dual intersection numbers of $\mathcal{W}$ (or Krein numbers of $\mathcal{W}$).
	}

	\only<3>{
		The intersection numbers of $\mathcal{W}$ are non-negative integers.

		\vspace{1em}

		The dual intersection numbers of $\mathcal{W}$ are non-negative real numbers.
	}

\end{frame}

\begin{frame}

\frametitle{Duality Mappings}

	\begin{definition}
		A duality mapping between (commutative) coherent algebras $\mathcal{W}$ and $\mathcal{X}$ of order $n$ is a linear isomorphism $T : \mathcal{W} \to \mathcal{X}$ such that:

		\begin{itemize}
		  	\item $T(M N) = T(M) \circ T(N) $ for all $M, N \in \mathcal{W}$.
		  	\item $T (M \circ N) = \frac{1}{n} T(M) T(N)$ for all $M, N \in \mathcal{W}$.
		\end{itemize}

  	\end{definition}


\end{frame}

\begin{frame}

\frametitle{Dual Coherent Algebras}

	There is a duality mapping between coherent algebras $\mathcal{X}$ and $\mathcal{Y}$ of dimension $d$
	if and only if:
		\begin{enumerate}
			\only<1->{
				\item There is a character table $P$ of $\mathcal{X}$ and dual character table $Q$ of $\mathcal{Y}$ such that $\overline{P} = Q$.
			}

			\only<2->{

				\item $p_{i,j}(k) = q_{i,j}(k)$ for all $1 \leq i,j,k \leq d$,

				\vspace{0.5em}

				where $\{ p_{i,j}(k) : 1 \leq i,j,k \leq d \}$ are the intersection numbers of $\mathcal{X}$ 

				\vspace{0.5em}

				and $\{ q_{i,j}(k) : 1 \leq i,j,k \leq d \}$ are the dual intersection numbers of $\mathcal{Y}$.

			}
		\end{enumerate}

\end{frame}

\begin{frame}

\frametitle{Metric and Cometric Coherent Algebras}

	\only<1-3>{ 

		A (commutative) coherent algebra $\mathcal{W}$ is said to be metric if $\mathcal{W} = \operatorname{span} \{ I, A, A^{2}, \cdots \}$
		for some Schur-primitive matrix $A \in \mathcal{W}$.

	}

	\only<2-3>{

		\vspace{1em}

		Dually, a (commutative) coherent algebra $\mathcal{W}$ is said to be cometric if $\mathcal{W} = \operatorname{span} \{ J, E, E^{\circ 2}, \cdots \}$
		for some primitive matrix $E \in \mathcal{W}$.

	}

	\only<3>{

		\vspace{1em} 
		If there is a duality mapping on a coherent algebra $\mathcal{W}$, then $\mathcal{W}$ is metric if and only if it is cometric.
	}

	\only<4->{
		Coherent algebras that are both metric and symmetric can be identified with the adjacency algebras of distance-regular graphs.
	}

	\only<5->{

		\vspace{1em}

		The eigenvalue multiplicities of a distance-regular graph are the valencies of its distance-graphs if there is a duality mapping
		on its adjacency algebra.

	}

\end{frame}

\begin{frame}

\frametitle{Self-Dual Coherent Algebras}

	\only<1>{
		\begin{definition}
		A coherent algebra $\mathcal{W}$ of order $n$ is said to be self-dual if there is
		a duality mapping $T : \mathcal{W} \to \mathcal{W}$ on $\mathcal{W}$ such that:
			\begin{itemize}
				\item $T^{-1}(M) = \frac{1}{n} T(M)$ for all $M \in \mathcal{W}$.
				\item $T(\lambda M) = \overline{\lambda} \ T(M)$ for all $\lambda \in \mathbb{C}$ and $M \in \mathcal{W}$.
			\end{itemize}
		\end{definition}

	}

	\only<2>{
		A coherent algebra $\mathcal{W}$ is self-dual if and only if there is a 
		character table $P$ for $\mathcal{W}$ with dual character table $\overline{P}$.
	}


\end{frame}

\begin{frame}

\frametitle{Strongly-Regular Graphs}

	\only<1>{

		Suppose that $G$ is a (connected) strongly-regular graph of order $n$ and valency $k$.

		\vspace{1em}

		$\implies P := \begin{pmatrix} 1 & 1 & 1 \\ k & \lambda & \mu \\ {n - 1 - k}& {-1 - \lambda} & {-1 - \mu} \end{pmatrix}$

		\vspace{0.5em}

		is a character table for $\mathcal{A}$ for some $\lambda, \mu \in \mathbb{R}$

		\vspace{1em}
	
		and $Q := \begin{pmatrix} 
			1 & 1 & 1 \\
			j & \left ( \frac{j}{k} \right ) \lambda & \left ( \frac{j}{n - 1 - k} \right ) \left ( -1 - \lambda \right ) \\
			{n - 1 - j} & \left ( \frac{n - 1 - j}{k} \right ) \mu &  \left ( \frac{n - 1 - j}{n - 1 - k} \right ) \left ( -1 - \mu \right ) \end{pmatrix}$

		\vspace{0.5em}

		is the dual character table of $\mathcal{A}$ relative to $P$ for some $j \in \mathbb{N}$.

	}


\end{frame}

\begin{frame}

\frametitle{Self-Dual Strongly-Regular Graphs}

	\only<1>{

		\begin{theorem}
			Suppose that $G$ is a strongly-regular graph of order $n$ and valency $k$. Then $G$ is self-dual 
			if and only if:

				\[ \left ( \lambda - \mu \right )^{2} = n \]

			with $\{k, \lambda, \mu \}$ the distinct eigenvalues of $G$.
		\end{theorem}

	}

	\only<2>{

		\begin{theorem}
			A strongly-regular graph $G$ is self-dual if and only if $G$ is either:
				\begin{enumerate}
					\item A conference graph.
					\item An $\operatorname{SRG} \left ( m^{2}, (m - 1) \gamma, (\gamma - 1)(\gamma - 2) + (m - 2), \gamma( \gamma - 1) \right )$ for some $m, \gamma \in \mathbb{N}$.
				\end{enumerate}
		\end{theorem}

	}

	\only<3>{

		\begin{theorem}
			A strongly-regular graph is self-dual if and only if the multiplicities of its eigenvalues are the valencies of 
			its distance-graphs.
		\end{theorem}

	}


\end{frame}

\begin{frame}

\frametitle{Self-Dual Distance-Transitive Graphs}

	\begin{example}
		The Hamming graphs are self-dual.
	\end{example}

\end{frame}

\begin{frame}

	\begin{center} 
		To be continued ...
	\end{center}

\end{frame}

\end{document}
