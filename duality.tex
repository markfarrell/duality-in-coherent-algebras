\documentclass[notheorems]{beamer}

\usetheme{Warsaw}
\useoutertheme{miniframes}
\useinnertheme{rectangles}
\usecolortheme{albatross}

\usepackage[utf8]{inputenc}

\usepackage{amsmath}
\usepackage{amsthm}

\theoremstyle{definition} % insert bellow all blocks you want in normal text
\newtheorem{definition}{Definition}[section] % to number according to section

%Information to be included in the title page:
\title{Duality in Coherent Algebras}
\author{Mark Farrell}
\date{February 2018}

\begin{document}

\begin{frame}
\titlepage
\end{frame}

\begin{frame}

\frametitle{Coherent Algebras}

	\begin{definition}

		  A coherent algebra $\mathcal{W}$ of order $n$ is a subspace of $\operatorname{Mat}_{n \times n}(\mathbb{C})$ such that:

		  \begin{itemize}
		  	\item $M N$ and $M \circ N \in \mathcal{W}$ for all $M, N \in \mathcal{W}$.
		  	\item $M^{T} \in \mathcal{W}$ for all $M \in \mathcal{W}$.
		  	\item $I, J \in \mathcal{W}$.
		  \end{itemize}

  \end{definition}

\end{frame}

\begin{frame}

\frametitle{Duality Mappings}

	\begin{definition}
		A duality mapping between (commutative) coherent algebras $\mathcal{W}$ and $\mathcal{X}$ of order $n$ is a linear isomorphism $T : \mathcal{W} \to \mathcal{X}$ such that:

			\begin{itemize}
		  		\item $T(M N) = T(M) \circ T(N) $ for all $M, N \in \mathcal{W}$.
		  		\item $T (M \circ N) = \frac{1}{n} T(M) T(N)$ for all $M, N \in \mathcal{W}$.
			\end{itemize}

  	\end{definition}

\end{frame}

\begin{frame}

\frametitle{Primitive Matrices}

	\begin{definition}
		An element $M \in \mathcal{W}$ of a coherent algebra $\mathcal{W}$ is said to be primitive if $M^{2} = M$ and $M N \in \operatorname{span} \{ M \}$ for all $N \in \mathcal{W}$.
	\end{definition}

\end{frame}

\begin{frame}

\frametitle{Schur-Primitive Matrices}

	\begin{definition}
		Dually, an element $M \in \mathcal{W}$ of a coherent algebra $\mathcal{W}$ is said to be Schur-primitive if $M^{\circ 2} = M$ and $M \circ N \in \operatorname{span} \{ M \}$ for all $N \in \mathcal{W}$.
	\end{definition}

\end{frame}

\begin{frame}

\frametitle{Dual Bases}


	\only<1->{Every coherent algebra is spanned by its set of Schur-primitive matrices.} 

	\vspace{1em}

	\only<2>{Dually, every commutative coherent algebra is spanned by its set of primitive matrices as well.}

\end{frame}

\begin{frame}



	Suppose that $\mathcal{W}$ is a commutative coherent algebra of order $n$.

	\vspace{1em}

	Let $\Gamma(\mathcal{W})$ denote the set of Schur-primitive matrices in $\mathcal{W}$.

	\vspace{1em}

	Dually, let $\Lambda(\mathcal{W})$ denote the set of primitive matrices in $\mathcal{W}$.

	\vspace{1em}

	Suppose that $\{ A_{i} : 1 \leq i \leq d \}$ and $\{ E_{i} : 1 \leq i \leq d \}$ are orderings 
	on $\Gamma(\mathcal{W})$ and $\Lambda(\mathcal{W})$, letting $d := \operatorname{dim}(\mathcal{W})$.

\end{frame}

\begin{frame}

\frametitle{Character Tables and Dual Character Tables}

	\only<1->{The matrix $P \in \operatorname{Mat}_{d \times d}(\mathbb{C})$ such that $A_{i} = \sum\limits_{j = 1}^{d} P_{i, j} E_{j}$
	is said to be the character table of $\mathcal{W}$ (under these orderings).}

	\only<2->{
		\vspace{1em}

		Dually, the matrix $Q \in \operatorname{Mat}_{d \times d}(\mathbb{C})$ such that $E_{i} = \sum\limits_{j = 1}^{d} \left ( \frac{1}{n} Q_{i, j} \right ) A_{j} $
		is said to be the dual character table of $\mathcal{W}$ (under these orderings).

	}

	\only<3->{
		\vspace{1em}

		We have that $P^{-1} = \frac{1}{n} Q$ and hence $P Q = n I$.
	}

\end{frame}

\begin{frame}

\frametitle{Intersection Numbers and Dual Intersection Numbers}

	\only<1>{
		There is a set of numbers $\{ p_{i,j}(k) : 1 \leq i,j,k \leq d \}$ such that $A_{i} A_{j} = \sum\limits_{k = 1}^{d} p_{i, j}(k) A_{k}$ for all $ 1 \leq i, j \leq d$.

		\vspace{1em}

		These are said to be the intersection numbers of $\mathcal{W}$.
	}

	\only<2>{
		Dually, there is a set of numbers $\{ q_{i,j}(k) : 1 \leq i,j,k \leq d \}$  such that $E_{i} \circ E_{j} = \sum\limits_{k = 1}^{d} q_{i, j}(k) E_{k}$ for all $ 1 \leq i, j \leq d$.

		\vspace{1em}

		These are said to be the dual intersection numbers of $\mathcal{W}$ (or Krein numbers of $\mathcal{W}$).
	}

	\only<3>{
		The intersection numbers of $\mathcal{W}$ are non-negative integers.

		\vspace{1em}

		The dual intersection numbers of $\mathcal{W}$ are non-negative real numbers.
	}

\end{frame}

\begin{frame}

\frametitle{Duality Mappings}

	There is a duality mapping between commutative coherent algebras $\mathcal{X}$ and $\mathcal{Y}$ of dimension $d$
	if and only if:
		\begin{enumerate}
			\item $p_{i,j}(k) = q_{i,j}(k)$ for all $1 \leq i,j,k \leq d$,

				\vspace{0.5em}

				where $\{ p_{i,j}(k) : 1 \leq i,j,k \leq d \}$ are the intersection numbers of $\mathcal{X}$ 

				\vspace{0.5em}

				and $\{ q_{i,j}(k) : 1 \leq i,j,k \leq d \}$ are the dual intersection numbers of $\mathcal{Y}$.

			\item There is a character table $P$ of $\mathcal{X}$ and dual character table $Q$ of $\mathcal{Y}$ such that $\overline{P} = Q$.
		\end{enumerate}


\end{frame}

\begin{frame}

\frametitle{Finite Groups}

	\only<1>{

		Suppose that $G$ is a finite group of order $n$.

		\vspace{1em}

		Let $L^{2}(G)$ denote the space of complex-valued functions on $G$.

		\vspace{1em}

		Define $M_{f} \in \operatorname{Mat}_{G \times G}(\mathbb{C})$ as $ \left ( M_{f} \right )_{x,y} := f(x^{-1} y)$ for all $x,y \in G$.

		\vspace{1em}

		We have that $M_{f} M_{g} = M_{f \star g}$ for all $f, g \in L^{2}(G)$ with convolution $f \star g$.

	}

	\only<2>{

		Additionally, define $A_{g} \in \operatorname{Mat}_{G \times G}(\mathbb{C})$ as $\left ( A_{g} ) \right )_{x,y} = xy^{-1}$ 

		for all $x,y \in G$ and $g \in G$.

		\vspace{1em}

		The coherent algebra $\mathcal{W}_{G} := \operatorname{span} \{ A_{g} : g \in G \}$

		\vspace{0.5em}

		$ = \{ M_{f} : f \in L^{2}(G) \}$ can be identified with the group ring of $G$.

		\vspace{1em}

		Refer to $W_{G}$ as the group coherent algebra of ${G}$.


	}

\end{frame}

\begin{frame}

\frametitle{Finite Abelian Groups}

	\only<1>{
		Assume that $G$ is abelian.

		\vspace{1em}

		Let $X(G)\subseteq L^{2}(G)$ denote the set of irreducible characters of $G$.

		\vspace{1em}

		Then $W_{G} = \operatorname{span} \left \{ \frac{1}{n} M_{\chi} : \chi \in X(G) \right \}$ 

		\vspace{0.5em}

		where $ \left \{ \frac{1}{n} M_{\chi} : \chi \in X(G) \right \}$ is the set of primitive matrices in $\mathcal{W}_{G}$.

	}

	\only<2>{
		A character table for $\mathcal{W}_{G}$ can be identified with a character table for $G$ in the usual sense.

		\vspace{1em}

		There is an ordering $\{ \chi_{x} : x \in G \}$ on $X(G)$ such that $\chi_{x}(y) = \chi_{y}(x)$ for all $x,y \in G$,

		\vspace{0.5em}

		since $G$ can be decomposed as a direct product of finite cyclic groups.

		\vspace{1em}

		There is then a character table $P$ for $\mathcal{W}_{G}$ such that $P \overline{P} = n I$,

		\vspace{0.5em}

		since $X(G)$ forms an orthonormal basis for $L^{2}(G)$ as well.

	}

\end{frame}

\begin{frame}

\frametitle{Fourier Transforms}

	\only<1>{
		\begin{definition}
			A Fourier transform on a (commutative) coherent algebra $\mathcal{W}$ of order $n$ is a duality mapping $T : \mathcal{W} \to \mathcal{W}$ such that:
			\begin{itemize}
				\item $T^{-1}(M) = \overline{ \frac{1}{n} T(M)}$ for all Schur-primitive matrices $M \in \mathcal{W}$.
			\end{itemize}
		\end{definition}
	}

	\only<2>{
		A Fourier transform on the group coherent algebra $\mathcal{W}_{G}$ of a finite abelian group $G$ can be identified with a Fourier transform on the space $L^{2}(G)$ of complex-valued functions on $G$.
	}

\end{frame}

\begin{frame}

\frametitle{Self-Dual Coherent Algebras}

	There is a Fourier transform on a (commutative) coherent algebra $\mathcal{W}$ if and only if there is a character table $P$ of $\mathcal{W}$
	such that dual character table of $\mathcal{W}$ is $\overline{P}$.

	\vspace{1em}

	A (commutative) coherent algebra is said to be self-dual if there is character table $P$ of $\mathcal{W}$ such that $\overline{P}$ is the dual character table of $\mathcal{W}$.

\end{frame}

\begin{frame}

\frametitle{More Self-Dual Coherent Algebras}

	\only<1-3>{

		The adjacency algebras of Hamming graphs are self-dual coherent algebras.

	}

	\only<2-3>{

		\vspace{1em}

		The tensor product $\mathcal{X} \otimes \mathcal{Y} := \{ M \otimes N : M \in \mathcal{X}, N \in \mathcal{Y} \}$ 

		\vspace{0.5em} 

		of coherent algebras $\mathcal{X}$ and $\mathcal{Y}$ is self-dual if there is a duality mapping betwen them.

	}

	\only<3>{

		\vspace{1em}

		There is a duality mapping on the adjacency algebra of every self-complementary strongly-regular graph.

	}

\end{frame}

\begin{frame}

\frametitle{Metric and Cometric Coherent Algebras}

	\only<1-3>{ 

		A (commutative) coherent algebra $\mathcal{W}$ is said to be metric if $\mathcal{W} = \operatorname{span} \{ I, A, A^{2}, \cdots \}$
		for some Schur-primitive metric $A \in \mathcal{W}$.

	}

	\only<2-3>{

		\vspace{1em}

		Dually, a (commutative) coherent algebra is said to be cometric if $\mathcal{W} = \operatorname{span} \{ J, E, E^{2}, \cdots \}$
		for some primitive matrix $E \in \mathcal{W}$.

	}

	\only<3>{
		A self-dual coherent algebra is metric if and only if it is cometric.
	}

	\only<4->{
		Coherent algebras that are both metric and symmetric can be identified with the adjacency algebras of distance-regular graphs.
	}

	\only<5->{

		\vspace{1em}
		
		The eigenvalue multiplicities of a self-dual distance-regular graph are the valencies of its distance-graphs.
	}


\end{frame}

\begin{frame}

Is it possible to classify the self-dual distance-regular graphs?

\end{frame}

\end{document}
